\documentclass[dvipsnames,a4paper,11pt]{article}

\usepackage[utf8]{inputenc}
\usepackage[T1]{fontenc}
\usepackage{a4wide}
\usepackage{amsmath}
\usepackage{todonotes}
\usepackage{xcolor}
\usepackage{amsthm}
\usepackage{amssymb}
\usepackage{accents}


\newtheorem{thm}{Theorem}
\newtheorem{prop}{Propery}

\theoremstyle{definition}
\newtheorem{defn}{Definition}

\theoremstyle{remark}
\newtheorem{rmq}{Remark}

\theoremstyle{remark}
\newtheorem{ex}{Example}

\newcommand{\tiph}[1]{\todo[inline,color=Orchid]{#1}}
\newcommand{\pimprenelle}[1]{\todo[inline, color=ForestGreen]{#1}}
\newcommand{\jeffin}[1]{\todo[inline, color=BrickRed]{#1}}
\newcommand{\jeff}[1]{\todo[color=BrickRed]{#1}}


\title{Multiplex stream graphs}
\author{Pimprenelle Parmentier, Tiphaine Viard, \ldots}

\begin{document}
    \maketitle

    \section{Multilayer stream graphs}

	\subsection{Definition}
    We define a multiplex stream graph as a tuple $M = (T,T_M,V,W_M,E_M,{\cal L})$, such that $T$ and $V$ are, respectively, a time interval and a set of nodes, just like for stream graphs \cite{stream}.
    As for multilayer graphs \cite{mlkiv}, ${\cal L}$ is a set of $d$ {\em aspects}; ${\cal L} = \{L_i\}_{i=1}^d$, and an element $\alpha_i$ of $L_i$ is named {\em elementary layer}. An element of $L=L_1\times \dots \times L_d$ is named {\em layer}.

	$T_M$ is a set of intervals of $T$ so for that each layer $\alpha$, $T_{\alpha}$ is the interval at which the layer exists. For each $t$ in $T$, $\exists \alpha \in L | t \in T_{\alpha}$ (at any time, all the aspects are non-empty).

	$W_M \subseteq T\times V \times (L_1 \times \dots \times L_d)$ represents the existing points on layers with respect to time.

   $E_M \subseteq T\times V_M \otimes V_M$ are the links appearing with respect to time,  $V_M = \{ (u,\alpha), u\in V, \alpha \in L\}$. The elements of $V_M$ are named {\em node-layer}.

    Notice that just like in multiplex networks, nodes in time ({\em i.e.}, elements of $W_M$) can be present in an arbitrary number of layers.

    \pimprenelle{Note that $V$ can be seen as the aspect $L_0$ with $T_0 = T$ and that more generally it is just an aspect similar to the other ones. .... a discuter}

   	\pimprenelle{La représentation de la présence ou non des couches en fonction du temps par $T_M$ est un choix, que nous préférons pour le moment à celle $M_M$ des couples $(t,\alpha)$}

	\subsection{2 examples}
	\begin{ex}[Research world]\label{exresearch}
	We consider a set of people working in research.
	
	These people can work in one or more universities and/or companies, in one or more departments. They can work together on different ways : they can collaborate on a paper, supervise or being supervised. And their work on different subjects, linked together thanks to keywords.


	The multilayer stream graph is the following :
	\begin{itemize}
		\item the {\bf nodes V} are the researchers
		\item the {\bf aspects L} are : the university/companies , the departments, the subjects, the type of status of interaction (supervisor/ supervised / collaborator) 
		\item {\bf links} appear during an interaction between two researchers.
		\item all these parameters vary with time, the set $T$ is chosen arbitrarily
	\end{itemize}
	
	Question : which aspect is the more important in the spreading and creation of knowledge ?
	
	\end{ex}
	
	\begin{figure}
    	\centering
    	\includegraphics[width=15cm]{schemas/chercheurs.jpg}
    	\caption{An simplified scheme of a multiplex stream graph for example \ref{exresearch}. The nodes are the researchers R1,R2,R3, and the aspects are the Universities (U1,U2,U3), and the department (Maths, Computer sciences, Physics)}
    	\label{fig:chercheurs}
	\end{figure}
	

	\begin{ex}[Epidemics between species]
		We consider an set of individuals from different species, and a disease. This disease could be transmitted through different types of interaction : by blood, by  by respiratory tracts, digestive tracks, etc.
		
		The stream-multilayer graph is the following : 
		\begin{itemize}
			\item the {\bf nodes V} are the individuals
			\item the {\bf aspects L} are : the species, the type of possible interaction, the state of the individual (sane, infected, contagious, dead, recoved ... ) 
			\item {\bf links} appear when there is an interaction between two individuals.
			\item all these parameters vary with time, the set $T$ is chosen arbitrarily (from example from the start of the infection)
		\end{itemize}
	
		The question here could be : when and where do we have to act (cut some connection / remove some node-layer) to stop the spreading of the disease as soon as possible ?
	\end{ex}

	
	\begin{rmq}
		We notice that it is easy to make the multilayer as complex as we want. Three rules seem to be important to have a pertinent use of multiplex-stream graph :
		\begin{itemize}
			\item a set of nodes which could have various properties which can be "melted"
			\item different ways to interact between the nodes in function of their properties
			\item a time-dependency
		\end{itemize}
	\end{rmq}
	
	
   	\subsection{Substreams and operations over substreams}

    From this definition, let us define common streams of interest.

	For any layer $\alpha \in L_1 \times \dots \times L_d$, the {\em intra-layer} stream graph $S^{\alpha}$ is the stream $S^{\alpha}=(T^{\alpha},V^{\alpha},W^{\alpha},E^{\alpha})$ such that for all $t$ in T :
	\begin{itemize}
		\item $T^{\alpha} = T_{\alpha} \in T_M$
		\item $V^{\alpha} =\{(u,\alpha),u\in V\} \cap V_M$
		\item $W^{\alpha} =\{(t,u,\alpha),|u\in V, t \in T \} \cap W_M$
		\item $E^{\alpha} = \{(t,(u,\alpha),(v,\alpha))| t \in T,(u,v)\in V^2 \} \cap E_M$.
	\end{itemize}


	For any two layers $\alpha, \beta \in L_1\times \dots\times L_d$, the {\em inter-layer} stream graph is the stream $S^{(\alpha,\beta)} = (T, V^{\alpha,\beta},W^{\alpha,\beta},E^{\alpha,\beta})$ such that :
	\begin{itemize}
		\item $T^{\alpha,\beta}=T^{\alpha}\cap T^{\beta}$
		\item $V^{\alpha,\beta} = \{u,\gamma | u \in V, \gamma \in \{\alpha,\beta\} \} \cap V_M$
		\item $W^{\alpha,\beta}= \{(t,u,\gamma) | u \in V, \gamma \in \{\alpha,\beta\} \} \cap W_M$
	    \item $E^{\alpha,\beta} = \{(t,(u,\alpha),(v,\beta) | u,v \in V\} \cap E_M $
	\end{itemize}

	    Notice that for any layer $\alpha$, $S^{(\alpha)}$ is equivalent to $S^{(\alpha,\alpha)}$.



	The {\em intersection} of two substreams $S_1$ and $S_2$ is defined as follows :
	\[
		S' = S_1 \cap S_2 = (T_1\cap T_2, V_1 \cap V_2, W_1 \cap W_2, E_1\cap E_2)
	\]

	Notice that this intersection is a stream graph too.

	The {\em union} of two substreams $S_1$ and $S_2$ is defined as follows :
	\begin{align*}
		S' = S_1 \cup S_2 = (T', V_1 \cup V_2, W_1 \cup W_2, E' \})\\
		T' = [\min(T_1,T_2),\max(T_1,T_2)]\\
		E' = E_1 \cup E_2 
	\end{align*}

	Notice that the union of two sub stream graphs gives a stream graph.

	

	\subsection{Projections }

    The {\em coupling edges} are $E_C=\{(t,u,\alpha,v,\beta)\in E_M | u=v\}$.

    The {\em intra-layer edges} are $E_I = \{(t,u,\alpha,v,\beta) \in E_M | \alpha = \beta \}$

    The {\em inter-layer edges} are $\bar{E_I} = E_M\backslash E_I$.

   	The {\em multilayer graph at time t} $M_t$ is $M_t = (V_{M,t}, E_{M,t},V_t,{\cal L}_t)$ with :
    \begin{itemize}
		\item $V_{M,t} = \{(u,\alpha)| (t,u,\alpha)\in W_M\}$
		\item $E_{M,t} = \{(u,\alpha,v,\beta) | (t,u,\alpha,v,\beta) \in E_M\}$
		\item $V_t = {u | (t,u) \in V}$
		\item ${\cal L}_t = {L_i,t}_{i=1}^d, L_{i,t}=(\alpha)_i, t\in T_\alpha$
    \end{itemize}

    \begin{rmq}
    	We assume that $\forall (i,t), \; L_{(i,t)} \neq \emptyset$.
    \end{rmq}

    A {\em multiplex stream graph (MpSG)} is a graph in which the only inter-layer connections allowed are connections between the same point : $E_M = E_I \cup E_C$.

    The {\em multilayer graph induced} $M_I(M)$ of $M$ is the multilayer graph which gather all the points on all the layer who appear during $T$ and all the links in $E$.
    \begin{align*}
    	M_I(M) = (V_{M,I}, E_{M,I}, V_I,L_I)\\
    	V_{M,I} = \bigcup_{t\in T} V_{M,t}\\
    	E_{M,I} = \bigcup_{t\in T} E_{M,t}\\
    	V_I = \bigcup_{t\in T} V_t \\
    	L_I = \bigcup_{t\in T} L_t\\
    \end{align*}

    The {\em underlying stream } $S_U(M)$ of $M$ is $(T,V_M,W_M,E_M)$, and could be share into clusters corresponding to the different layers. This definition is a generalisation to the one used for simple multilayer graphs.

	\begin{prop}
		\[
			\bigcup_{(\alpha,\beta) \in L^2} S^{(\alpha,\beta)} = S_U(M)
		\]
	\end{prop}
	\begin{proof}
		\begin{align*}
			S=\bigcup_{(\alpha,\beta) \in L^2} S^{(\alpha,\beta)}\\
			S=(T^{*},V^{*},W^{*},E^{*})\\
			S_U(M) = (T_U,V_U,W_U,E_U)			
		\end{align*}		 
	
	
	\begin{itemize}
		\item $T=T_U$ by definition of $S_U$. Let's take $t\in T$. By definition of multilayer stream graphs, $\exists \alpha \in L | t\in T_{\alpha}=T^{\alpha,\alpha}$. So $t \in [\underset{(\alpha,\beta) \in L^2}{\min}(T^{\alpha,\beta}),\underset{(\alpha,\beta) \in L^2}{\max}(T^{\alpha,\beta})]=T^{*}$. $T_U \in T^{*}$. 
	Moreover, $\underset{(\alpha,\beta) \in L^2}{\min}(T^{\alpha,\beta}) \geq \min(T)$ 
	and $\underset{(\alpha,\beta) \in L^2}{\max}(T^{\alpha,\beta}) \leq \max(T)$. 
	So $\mathbf{T^{*} = T}$.
		\item  $V_M = V_U$ by definition of $V_U$.$V^{*}=\bigcup_{(\alpha,\beta) \in L^2} V^{\alpha,\beta} \subset V = V_M$.
	Let $(v,\alpha) \in V_U$. So $(v,\alpha) \in V^{\alpha,\alpha} \in V^{*}$.$V^{*}=V_U$.$\mathbf{V^{*}=V_U}$.
		\item $W_U=W_M$ and $W^{*}=\bigcup_{\alpha,\beta \in L^2} W^{\alpha,\beta} \subset W_M$. So as for all $ (t,u,\alpha)$ in $W_M$, $(t,u,\alpha)$ belongs to $W^{\alpha,\alpha}$, $W_M$ is a subset of $W^{*}$. So $\mathbf{W^{*}=W_M}$.
		\item $E_U=E_M$ and $E^{*}=\bigcup_{\alpha,\beta \in L^2}(E^{\alpha,\beta})$ by definition. So as $E^{\alpha,\beta}$ is a subset of $E_M$ for every couple of layers $\alpha,\beta$, $E^{*} \subset E_M$. 
		$\forall e =(t,u,\alpha,v,\beta) \in E_M$, $e \in E^{\alpha,\beta}$. So $\mathbf{E_M=E^{*}}$.
	\end{itemize}		
	\end{proof}



    The {\em contribution} of one node-layer in the MPG is $n_{v,\alpha} = \frac{|T_{v,\alpha}|}{|T|}$. The {\em number of node-layers} is the sum of the contributions $N(M) = \underset{(u,\alpha)\in V_M}{\sum} n_{(u,\alpha)} = \frac{|W_M|}{|T|}$.

	The {\em overlap presence } of two nodes-layers $(u,\alpha)$ and $(v,\beta)$ measures the rate of simultaneous presence of two node-layers :
	\begin{align*}
		\mathbb{U}((u,\alpha),(v,\beta))=\mathbb{P}( t \in T_{u,\alpha} \cap T_{v,\beta} | t \in T_{u,\alpha} \cup T_{u,\beta}) \\
		= \frac{|T_{u,\alpha}\cap T_{u,\beta}|}{|T_{u,\alpha}\cup T_{u,\beta}|}
	\end{align*}

	The {\em overlap presence} of two layers $\alpha$ and $\beta$ is defined on the same pattern :
	\begin{align*}
		\mathbb{U}(\alpha,\beta) = \frac{|T_{\alpha}\cap T_{\beta}|}{|T_{\alpha}\cup T_{\beta}|}
	\end{align*}

    The {\em node-layer uniformity} measures the rate of simultaneous presences of node-layers in the multilayer-stream graph :
    \[
    	\mathbb{U}(M) = \frac{\sum_{(u,\alpha),(v,\beta) \ V_M \otimes V_M}{|T_{(u,\alpha)} \cap T_{(v,\beta)}|}}{\sum_{(u,\alpha),(v,\beta) \ V_M \otimes V_M}{|T_{(u,\alpha)}\cup T_{(v,\beta)}|}}
    \]



	\begin{rmq}
		Note that this concept can be used for sub-stream graphs of any type.
	\end{rmq}

	When $\mathbb{U}=1$, we say that the multilayer stream graph is uniform.





    \section{Intrication (Entanglement)}

	************
    Intrication, extensions from stream graphs (inter-density, intra-density, etc.)

    Special case when the multiplex stream can be expressed in clusters of the stream graph.
    ************



    \subsection{Density}
	In classical graphs (unweighed and non directional), the density $d(G)$ measures how much the vertexes are connected.
		\[
			d(G) = \frac{|E|}{|V\otimes V|} = \frac{2\times |E|}{|V|(|V|-1)}
		\]
	In stream graphs, the density measures too the "rate of connectivity", or the probability that, at a randow time when two point exist, they are connected.

		\begin{align*}
			\delta_s(G) = & \mathbb{P}((t,u,v)\in E| (t,u),(t,v) \in W) \\
			 =  & \frac{\sum_{uv \in V \otimes V}{|T_{uv}|}}{\sum_{uv \in V\otimes V}{|T_u\cap T_v|}}= \frac{\int_{t\in T}{|E_t|dt}}{\int_{t\in T}{|V_t\otimes V_t|dt}}
		\end{align*}

	We can then define the density of the multilayer graph. Notice first that the density measures the rate of connectivity, and so depending to the type of system we study, the definition can change. If some connections are forbidden, we will ignore it in the density. We name $C \in V_M\times V_M$ the set of possible connections in the mutlilayer.
	\[
		\delta_M (M) = \frac{|E_M|}{|\{\text{"possible connections"}\}|}
	\]

	The generalisation to multilayer stream graphs is then natural.
	\[
		\delta_M (M) = \frac{\int_{t\in T}|E_M,t|}{\int_{t\in T}|C|} = \frac{\sum_{(u,\alpha)(v,\beta) \in V_M \otimes V_M}|T_{(u,\alpha)(v,\beta)}|}{\sum_{(u,\alpha)(v,\beta) \in C}|T_{(u,\alpha)}\cap T_{(v,\beta)}|}
	\]

	For example, in a multiplex stream graph, $C=\{(t,u,\alpha),(t,v,\beta))| t\in T_{u,\alpha} \cap T_{u,\beta}, u=v \text{or} \alpha = \beta)\}$ :

	\[
		\delta_M (M) = \frac{|E_M|}{|C|}= \frac{\sum_{(u,\alpha)(v,\beta) \in (V_M \otimes V_M)} |T_{(u,\alpha)(v,\beta)}|}{\sum_{\alpha \in L}\sum_{(u,v) \in V\otimes V}|T_{u,\alpha} \cap T_{v,\alpha}|+ \sum_{u \in V } \sum_{(\alpha,\beta) \in L \otimes L}|T_{u,\alpha}\cap T_{u,\beta}|}
	\]

    \section{Isomorphism in multiplex stream graphs}
    \paragraph{}
    In classical graphs, $G$ and $G'$ are isomorphs if we can find $f:V \rightarrow V'$ bijective such as $G_f=G'$,$G_f=(f(V),E_f), E_f={(f(u),f(v)), (u,v) \in E}$.
    \paragraph{}
    Given a stream graph $S=(T,V,W,E)$ and $S'=(T,V,W,E)$ we define $f=(f_T,f_V,f_V,f_E)$ such that:
    \begin{itemize}
    	\item $f_T(t)= t + \beta, \alpha > 0$
    	\item $f_V:V\rightarrow V'$ is a bijection.
    	\item $f_W((t,u))=(f_T(t),f_V(u)) \forall (t,u) \in W$
    	\item $f_E((t,u,v))=(f_T(t),f_V(u),f_V(v)) \forall (t,u,v) \in E$
    	\item $f(S)=(f_T(T),f_V(V),f_W(W),f_E(E))$
    \end{itemize}

    Two stream graphs $S$ and $S'$ are {\em isomorphs } if we can find $f$ such as $f(S)=S'$.
    Two stream graphs are {\em shifted} if we can find $f$ with $f_V=\text{Id}$ and $f(S)=S'$.

	\begin{rmq}
		We can also generalise again with $f_T(t)=\alpha t$ . We say that one graph is expanded/contracted with respect to the other.
	\end{rmq}



    \pimprenelle{This notion could be usefull for example to find some patterns in a music sheet. The theme could for example be done at different times with different instruments, and at different tones, and different tempis. (The use of multilayer could so be justified)}

    \nocite{*}
    \bibliographystyle{plain}
	\bibliography{main}

\end{document}
