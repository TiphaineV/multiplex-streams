\documentclass[dvipsnames,a4paper,11pt]{article}

\usepackage[utf8]{inputenc}
\usepackage[T1]{fontenc}
\usepackage{a4wide}
\usepackage{amsmath}
\usepackage{todonotes}
\usepackage{xcolor}

\newcommand{\tiph}[1]{\todo[inline,color=Orchid]{#1}}
\newcommand{\pimprenelle}[1]{\todo[inline, color=ForestGreen]{#1}}


\title{Multiplex stream graphs}
\author{Pimprenelle Parmentier, Tiphaine Viard, ...}

\begin{document}
    \maketitle

    \section{Multiplex stream graphs}

    We define a multiplex stream graph as a tuple $M = (T_M, V_M, W_M, E_M, T, V, {\cal L})$, such that $T_M \subseteq T$, $V_M \subseteq V \times {\cal L}$, $W_M \subseteq T_M\times V_M$ and $E_M\subseteq T_M \times V_M \otimes V_M$.
    $T$ and $V$ are, respectively, a time interval and a set of nodes, just like in stream graphs.
    We then define {\em layers}; a multiplex stream graph can have an arbitrary number $d$ of aspects, and we define ${\cal L}$, as $\{L_i\}_{i=0}^{d}$ the set of layers, such that there is one layer for each $i=0..d$.
    Notice that just like in multiplex networks, nodes in time ({\em i.e.}, elements of $W_M$) can be present in an arbitrary number of layers.
    
    \tiph{Je ne suis pas sûre que l'on aie besoin de $V_M$, en fait; toute l'info de présence des n\oe{}uds dans les couches peut être contenue dans $W_M$ ?}
    \pimprenelle{Je ne sais pas, ça dépend de si on considère qu'un n\oe{}ud qui n'apparait à aucun temps t dans une couche existe dans cette couche ou pas : on pourrait avoir un n\oe{}ud qui apparait dans $V_M$ et pas dans $W_M$ même si c'est tordu ... et puis à ce moment là même dans un stream graph classique $S=(T,V,W,E)$, est-ce qu'on pourrait pas se dire que de la même façon que V n'est pas "utile" en terme d'informations supplémentaires ?}
    \pimprenelle{Je ne vois pas bien ce que représente la variable $T_M$ par rapport à la variable $T$ ? est-ce que les layers elle-mêmes seraient dépendantes du temps ? 
    Proposition : $T_M$ is a set of $T_i \subseteq T$.}
    
	\pimprenelle{Il y a une subtilité sur les layers : an aspect $(i)$ is a set $L_i$ of  elementary layers $\alpha^{i}_1,\dots,\alpha^{i}_n$
    		 a layer $\alpha$ is an combination of elementary layer of each aspect : $\alpha \in L_1\times\dots\times L_d$
   	}
   	
   	
    From this definition, let us define common streams of interest.
    For any layer $i \in {\cal L}$, the {\em intra-layer} stream graph $S^{(i)}$ is the stream such that ...
	
	\pimprenelle{Suggestion :}
	For any layer $\alpha \in L_1 \times \dots \times L_d$, the {\em intra-layer} stream graph $S^{\alpha}$ is the stream $S^{\alpha}=(T,V^{\alpha},W^{\alpha},E^{\alpha})$ such that for all $t$ in T :
	\begin{itemize}
		\item $V^{\alpha}$ is the set of nodes $(u,\alpha) \in V_M$
		\item $W^{\alpha}$ is the set of $(t,u,\alpha)$ included in $W_M$. 
		\item $E^{\alpha}$ is the set of links $(t,(u,\alpha),(v,\alpha))$ included in $E_M$.		
	\end{itemize}
	
	\pimprenelle{fin}	
	
    For any two layers $i,j\in {\cal L}$, the {\em inter-layer} stream graph $S^{(i,j)}$ ...
	\pimprenelle{Suggestion}
	For any two layers $\alpha, \beta \in L_1\times \dots\times L_d$, the {\em inter-layer} stream graph is the stream $S^{\alpha,\beta} = (T, V^{(\alpha,\beta)},W^{(\alpha,\beta)},E^{(\alpha,\beta)})$ such that :
	\begin{itemize}
		\item $V^{(\alpha,\beta)} = \{(u,\gamma) | u \in V, \gamma \in \{\alpha,\beta\} \} \cap V_M$
		\item $W^{(\alpha,\beta)}= \{(t,u,\gamma) | u \in V, \gamma \in \{\alpha,\beta\} \} \cap W_M$
		\item $E^{(\alpha,\beta)} = \{(t,(u,\alpha),(v,\beta) | u,v \in V\} \cap E_M $
	\end{itemize}
	\pimprenelle{fin}
    Notice that for any layer $\alpha$, $S^{(\alpha)}$ is equivalent to $S^{(\alpha,\alpha)}$.
	
	
	
    The {\em coupling edges} are $E_C=\{(t,u,\alpha,v,\beta)\in E_M | u=v\}$.
    
    The {\em intra-layer edges} are $E_I = \{(t,u,\alpha,v,\beta) \in E_M | \alpha = \beta \}$
    
    The {\em inter-layer edges} are $\bar{E_I} = E_M\backslash E_I$.
    
    Given a time $t\in T$, $M_t = (V_{M,t}, E_{M,t},V_t,\cal{L}_t)$ with : 
    \begin{itemize}
		\item $V_{M,t} = \{(u,\alpha)| (t,u,\alpha)\in W_M\}$
		\item $E_{M,t} = \{(u,\alpha,v,\beta) | (t,u,\alpha,v,\beta) \in E_M\}$
		\item $V_t = {u | (t,u) \in V}$
		\item ${\cal L}_t = \{L_i,t}_{i=1}^d \Alpha$
    \end{itemize}

	\pimprenelle{Il faut qu'on choisisse comment les layers varient au cours du temps. On peut choisir d'avoir un ensemble d'aspects $L_i$ fixe et faire varier seulement les couches élémentaires $\alpha_i$. Mais que se passe-t-il à ce moment là si toutes les couches elem d'un aspect disparaissent en même temps ? Le cas d'un aspect vide n'est pas discuté dans l'article mais il parait problématique. On n'a pas envie de faire disparaitre un aspect pour des raisons pratiques évidentes.}
	
    Intrication, extensions from stream graphs (inter-density, intra-density, etc.)

    Special case when the multiplex stream can be expressed in clusters of the stream graph.
    
       
    \section{Isomorphism in multiplex stream graphs}
    \paragraph{}
    In classical graphs, $G$ and $G'$ are isomorphs if we can find $f:V \rightarrow V'$ bijective such as $G_f=G'$,$G_f=(f(V),E_f), E_f={(f(u),f(v)), (u,v) \in E}$.
    \paragraph{}
    Given a stream graph $S=(T,V,W,E)$ and $S'=(T,V,W,E)$ we define $f=(f_t,f_v,f_w,f_e)$ :
    \begin{itemize}
    	\item $f_t(t)=\alpha t + \beta, \alpha > 0$
    	\item $f_v:V\rightarrow V'$ is a bijection.
    	\item $f_w((t,u))=(f_t(t),f_u(u)) \forall (t,u) \in W$
    	\item $f_e((t,u,v))=(f_t(t),f_v(u),f_v(v)) \forall (t,u,v) \in E$
    	\item $f(S)=(f_t(T),f_v(V),f_w(W),f_e(E))$
    \end{itemize}
    
    Two stream graphs $S$ and $S'$ are {\em isomorphs } if we can find $f$ such as $f(S)=S'$.
    Two stream graphs $S$ and $S'$ are {\em sychnronised} if we can find $f$ with $f_t=Id$ and $f(S)=S'$.
    Two stream graphs are {\em shifted} if we can find $f$ with $f_t(t)=t+\beta$ and $f(S)=S'$.
    
    
\end{document}
