\documentclass[11pt,a4paper]{article}
\usepackage[utf8]{inputenc}
\usepackage{amsmath}
\usepackage{amsfonts}
\usepackage{amssymb}
\usepackage{graphicx}
\usepackage[left=2cm,right=2cm,top=2cm,bottom=2cm]{geometry}
\author{Pimprenelle Parmentier}

\title{Flots de liens multicouches}
\begin{document}
\begin{titlepage}

\noindent
\textsc{Ecole polytechnique}\\
PROMOTION X2016 \\
MASTER: Mathématiques appliquées\\
PARMENTIER Pimprenelle

\vspace{3cm}
\begin{center}
\textsc{\Large Rapport de stage}
\vspace{1cm}
\hrule % Horizontal line
\vspace{0.4cm}
{\huge \bfseries Flots de liens multicouches \par}\vspace{0.4cm} % Thesis title
\hrule 
\vspace{1cm}
\textsc{\Large Rapport non confidentiel}
\vspace{4cm} % Horizontal line
 
\end{center}

\noindent
\textit{Option:} Département de Mathématiques appliquées\\
\textit{Champ:} Etude de graphes\\
\textit{Enseignant référent:} Xavier ALLAMIGEON\\
\textit{Tuteur de stage dans l'organisme:} Tiphaine VIARD\\
\textit{Dates du stage:} 8 avril 2019 - 23 aout 2019\\
\textit{Adresse de l'organisme:}\\
RIKEN AIP\\
Nihonbashi 1-chome Mitsui,\\
Building, 15th floor,\\
1-4-1 Nihonbashi,\\
Chuo-ku, Tokyo\\
103-0027, Japan\\
\end{titlepage}

\section{Etat de l'art}

\subsection{Les graphes multicouches}
\subsection{Les stream graphs (ou flots de liens)}



\section{Présentation d'un nouvel objet: le flot de liens multicouches}
\subsection{Motivations}
\subsection{Définition}
\subsection{Extraction de sous-graphes}
\subsection{Mesures}



\section{Application à des données concrètes}

\subsection{Structure de données et organisation du code}
\subsection{Classes préparatoires}
\subsection{Star Wars}
\subsection{Avions ?}

\section{Conclusions et perspectives}


\end{document}